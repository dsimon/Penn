% CSE 371 Final Writeup
%
% Derron Simon

\documentstyle[11pt]{article}
\begin{document}


\centerline{\LARGE \bf 8085 Minimal System Overview}
\centerline{\Large \bf Locker \#75}
\centerline{\Large \bf Sam Alcoff \& Derron Simon}


\section{System Design}

Our Initial board layout proved to be sufficient for the remainder of the project.  We
chose a design with the Vcc and GND lines down the center of the board to facilitate easier chip
powering.  We also chose to put our support circuitry on the front left and locate the 8085
nearby to make the switches easier to reach and to cut down on the distance from the crystal to
the 8085's clock input.  At that point we decided to keep all our RAMs and ROMs together on one 
side, and to keep the UART and 8155 together on the other.

By Lab Assignment 3 we started the actual system design.  Since the 8085 begins execution
at location 0000H on power-up the 4k ROM was mapped to that address.  The decision was made to break the
8085's 64k of address space into eight 8k blocks.  This allowed us to use a 74LS383 decoder to
decode the top 3 address lines into 8 chip select lines.  The lower most line selects the 4k ROM,
which is socketed to allow upgrade to an 8k ROM.

Since the 8085 uses the lower 8 address lines for both addressing and data, an
8 bit latch was added to hold the address lines when data is on the bus.  The effect of this was 
to create a separate address and data bus.

For Lab Assigment 4 we added an 8155 and LED diagnostic display.  This was where we made our 
first real design decision: whether to use memory mapped I/O or not.  Memory mapped was chosen over
I/O mapped.  The reason for this is that we considered the I/O routines of the 8085 another possible 
pitfall and we already had the memory decoding logic in place.  We figured we could spare 8k of addressable
memory for the benefits it would bring in the ease of writing software.

The 8155's SRAM was simple to install.  Since the 8155 has both a CE and IO/M line, we took
the high line from the 3-to-8 decoder and connected it to CE, and the A12 line from the 8085 and
connected it to the IO/M line.  This had the benefit of splitting one 8k segment into two 4k 
segments, the upper 4k for I/O and the lower 4k from SRAM.

We initially tried to set up the 8155 timer to generate interrupts.  We were going to use
the 7.5 interrupt of the 8085 since it was the only interrupt with a latched input.  We though we
could send TIMEROUT to the 7.5 input.  This seemed like a good idea, but we realised we would be
using the TIMER to generate UART clocking later and the interrupts would just be another thing
in the way.  We also thought that we might need to generate a refresh cycle or do interrupt driven
I/O in the final project, so we decided to leave our options open.

When attaching the LED diagnostic display we considered simply tying the output lines of the
8155 to the pins of the LEDs and have port A control LED \#1 and PORT B control LED \#2.  This was
scrapped in favor of using only PORT A with two 4-to-8 7-segment decoders on the high and low
nibbles of PORT A.  This allowed us to leave PORT B free which benefitted us greatly in our final
project.

In Lab Assigment 5 an 8k SRAM and 8251 UART were added.  Since the 8K SRAM is pin compatible
with the 8K ROM socket, it was easily installed next to the ROM.  The CS line of the SRAM was mapped
to the second lowest output of the decoder, thus locating it in the 8k segment above the ROM.  Since
the SRAM is pin compatible it is possible to remove the SRAM and replace it with a ROM if the need
arises.

When wiring up the UART we ran into our first real problem.  Initially the C/D line was connected
directly to the AD0 line, not the buffered address lines.  This caused us to spend many hours trying
to determine why sometimes we got garbage and other times didn't.  We eventually found the problem
and fixed it easily.

\section{Final Project Design}

Our 8085 minimal system final project was a sound generator with joystick input.  We had wanted to
use an LCD display with joystick and make a game, but we asked a day late and all the LCD displays
were gone.

To make it easier to implement we added another 8155 timer.  This provided 2 more ports for joystick
input and output to the sound chip.  The sound chip was connected to PORT A of one 8155 and the
control lines of the sound chip were connected to PORT C of the same 8155.  By interfacing to the
sound chip through and 8155 we didn't have to worry about interfacing a non-8085 component to
to 8085 system bus - which served the purpose of isolating sound chip problems from the rest
of the minimal system.

The sound chip worked very straightforwardly.  We didn't have to design any amplification cirvuits
because we used RCA inputs on Sam's portable stereo, so all that was required was to connect the
chip to the ports and the output to an 1/8'' plug from Radio Shack.

The A/D converter we found in the stock room didn't work.  This proved a very big problem.  We 
couldn't finish the lab on time because the stockroom didn't have any replacement A/D converters.
That Sunday, 1 day late, Sam's dad brought us a single A/D converter.  We hooked up the converter
to read the input of an Apple II joystick (0 to 5 volts).

Upon testing the joystick we found that Apple Joystick didn't have the variable potentiometer
set up correctly.  We had to add a ground to one side of the potentionmeter to get a +5v to 0v
range of outputs. 

Since the A/D converter we found didn't have multiplexed input, we could only hook up the Y-axis
of the joystick.

\section{Final Project Usage}

Our final project had a menu with options to manually test all of the components and a ``go'' option
which looped continuously taking input from the A/D converter (ie.\ joystick) and changing the tone
of the sound output to match the 8 bit input of the A/D.

\end{document}







